\documentclass[11pt,a4paper,oneside]{report}
\usepackage{bpchem}
\usepackage{fontspec}
\usepackage{titlesec}
\usepackage{natbib}
\usepackage{url}

\setmainfont{Arial}
\titleformat{\chapter}{\bfseries}{\thechapter.}{0em}{}

\begin{document}
\renewcommand\contentsname{Inhaltsverzeichnis}
\tableofcontents


\chapter{Einleitung und Vorbetrachtung}

Glas also \BPChem{SiO\_2} ist einer der Wenigen Feststoffe die für sichtbares Licht durchlässig sind. Diese Facharbeit möchte rechnerisch aufzeigen, warum Glas das sichtbare Licht durchlässt.
\\Stoffe sind undurchlässig für Licht und andere Elektromagnetische Wellen, wenn sie Elektronen haben, die mit diesen reagieren könnten.
\\Zur Betrachtung dieses Umstandes müssen drei Effekte betrachtet werden. \cite{pape99}

\chapter{Beschreibung der physikalisch-chemischen Modelle}

\section{Bohrsches Atommodell}

\section{Photon-Elektron-Reaktion}

\chapter{Mathematischer Beweis}

\chapter{Schlussbetrachtung}



\appendix{Anhang}
\chapter{Quellen}
\section{Onlinequellen}
\bibliographystyle{plain}
\bibliography{Sources}

%Michael Ralph Pape [http://fam-pape.de/raw/ralph/studium/teilchenphysik/] (13.10.2015)


\end{document}