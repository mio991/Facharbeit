\documentclass[11pt,a4paper,oneside]{report}
\usepackage{bpchem}
\usepackage{fontspec}
\usepackage{titlesec}
\usepackage{natbib}
\usepackage{url}
\usepackage{amsmath}
\usepackage{amsfonts}

\setmainfont{Arial}
\titleformat{\chapter}{\bfseries}{\thechapter.}{0em}{}

\begin{document}
\renewcommand\contentsname{Inhaltsverzeichnis}
\tableofcontents


\chapter{Einleitung und Vorbetrachtung}

Glas also \BPChem{SiO\_2} ist einer der wenigen Feststoffe die für sichtbares Licht durchlässig sind. Diese Facharbeit möchte rechnerisch aufzeigen, warum Glas das sichtbare Licht durchlässt.
\\Stoffe sind undurchlässig für Licht und andere elektromagnetische Wellen, wenn sie Elektronen haben, die mit diesen reagieren könnten.
\\Zur Betrachtung dieses Umstandes müssen drei Effekte betrachtet werden. \cite{pape99}

\chapter{Beschreibung der physikalisch-chemischen Modelle}

\section{Modelle der Quantenmechanik}
In der Quantenmechanik wird angennommen, das alle Elementarteilchen sowohl Teilchen- als auch Wellencharakter haben. Auserdem haben auch die bis dahin nur als Feld beschriebenen Effekte Teilchencharakter zum Beispiel das elektromagnetische Feld das durch Photonen übertragen wird.
Für diese Arbeit ist alerdings nur ein Teil der Quanten relevant dies sind:

\begin{tabular}{|l|lr|} \hline
Quantum 			& Eigenschaften 	&  		\\  \hline
Elektron( $e^-$ ) 		& Masse: 			& $m_e$	\\
					& Ladung: 		& $-e$ 	\\ \hline
Proton($p^+$) 		& Masse: 			& $m_p$ 	\\
 					& Ladung: 		& $e$ 	\\ \hline 
Neutron($N$)			& Masse:			& $m_N$	\\ 
					& Ladung:		& $0$	\\ \hline
Photon($ \lambda $) 	& Masse: 			& $0$	\\ 
					&Ladung:			& $0$	\\ \hline
\end{tabular}
\subsection{Das Elektron}


\section{Orbitalmodell und erweitertes Orbitalmodell}
Das Bohrsche Atommodell beschreibt ein 

\section{Photon-Elektron-Reaktion(Photoeffekt)}
Der Photoeffekt oder auch lichtelektrischer Effekt beschreibt die Aufnahme eines Photons durch ein Elektron wobei dieses die Energie des Photons aufnimt.\cite{stroppe08}

\section{Compton-Effekt}

\section{Paarbildung}

\chapter{Mathematischer Beweis}

\chapter{Schlussbetrachtung}



\appendix{Anhang}
\chapter{Quellen}
\section{Onlinequellen}
\bibliographystyle{plain}
\bibliography{Sources}

\end{document}